\PassOptionsToPackage{unicode=true}{hyperref} % options for packages loaded elsewhere
\PassOptionsToPackage{hyphens}{url}
%
\documentclass[
]{book}
\usepackage{lmodern}
\usepackage{amssymb,amsmath}
\usepackage{ifxetex,ifluatex}
\ifnum 0\ifxetex 1\fi\ifluatex 1\fi=0 % if pdftex
  \usepackage[T1]{fontenc}
  \usepackage[utf8]{inputenc}
  \usepackage{textcomp} % provides euro and other symbols
\else % if luatex or xelatex
  \usepackage{unicode-math}
  \defaultfontfeatures{Scale=MatchLowercase}
  \defaultfontfeatures[\rmfamily]{Ligatures=TeX,Scale=1}
\fi
% use upquote if available, for straight quotes in verbatim environments
\IfFileExists{upquote.sty}{\usepackage{upquote}}{}
\IfFileExists{microtype.sty}{% use microtype if available
  \usepackage[]{microtype}
  \UseMicrotypeSet[protrusion]{basicmath} % disable protrusion for tt fonts
}{}
\makeatletter
\@ifundefined{KOMAClassName}{% if non-KOMA class
  \IfFileExists{parskip.sty}{%
    \usepackage{parskip}
  }{% else
    \setlength{\parindent}{0pt}
    \setlength{\parskip}{6pt plus 2pt minus 1pt}}
}{% if KOMA class
  \KOMAoptions{parskip=half}}
\makeatother
\usepackage{xcolor}
\IfFileExists{xurl.sty}{\usepackage{xurl}}{} % add URL line breaks if available
\IfFileExists{bookmark.sty}{\usepackage{bookmark}}{\usepackage{hyperref}}
\hypersetup{
  pdftitle={Методические указания по выполнению расчётно-графической работы в R},
  pdfauthor={Михаил Лысенко},
  pdfborder={0 0 0},
  breaklinks=true}
\urlstyle{same}  % don't use monospace font for urls
\usepackage{longtable,booktabs}
% Allow footnotes in longtable head/foot
\IfFileExists{footnotehyper.sty}{\usepackage{footnotehyper}}{\usepackage{footnote}}
\makesavenoteenv{longtable}
\usepackage{graphicx,grffile}
\makeatletter
\def\maxwidth{\ifdim\Gin@nat@width>\linewidth\linewidth\else\Gin@nat@width\fi}
\def\maxheight{\ifdim\Gin@nat@height>\textheight\textheight\else\Gin@nat@height\fi}
\makeatother
% Scale images if necessary, so that they will not overflow the page
% margins by default, and it is still possible to overwrite the defaults
% using explicit options in \includegraphics[width, height, ...]{}
\setkeys{Gin}{width=\maxwidth,height=\maxheight,keepaspectratio}
\setlength{\emergencystretch}{3em}  % prevent overfull lines
\providecommand{\tightlist}{%
  \setlength{\itemsep}{0pt}\setlength{\parskip}{0pt}}
\setcounter{secnumdepth}{5}
% Redefines (sub)paragraphs to behave more like sections
\ifx\paragraph\undefined\else
  \let\oldparagraph\paragraph
  \renewcommand{\paragraph}[1]{\oldparagraph{#1}\mbox{}}
\fi
\ifx\subparagraph\undefined\else
  \let\oldsubparagraph\subparagraph
  \renewcommand{\subparagraph}[1]{\oldsubparagraph{#1}\mbox{}}
\fi

% set default figure placement to htbp
\makeatletter
\def\fps@figure{htbp}
\makeatother

\usepackage{booktabs}
\usepackage[]{natbib}
\bibliographystyle{apalike}

\title{Методические указания по выполнению расчётно-графической работы в R}
\author{Михаил Лысенко}
\date{2020-04-11}

\begin{document}
\maketitle

{
\setcounter{tocdepth}{1}
\tableofcontents
}
\hypertarget{ux437ux430ux434ux430ux43dux438ux435-ux43dux430-ux440ux430ux441ux447ux451ux442ux43dux43e-ux433ux440ux430ux444ux438ux447ux435ux441ux43aux443ux44e-ux440ux430ux431ux43eux442ux443}{%
\chapter{Задание на расчётно-графическую работу}\label{ux437ux430ux434ux430ux43dux438ux435-ux43dux430-ux440ux430ux441ux447ux451ux442ux43dux43e-ux433ux440ux430ux444ux438ux447ux435ux441ux43aux443ux44e-ux440ux430ux431ux43eux442ux443}}

Федеральной службой государственной статистики РФ ежеквартально проводятся бюджетные обследования домашних хозяйств по всем регионам России об условиях проживания и уровне благосостояния населения. Необходимо проанализировать данные обследования о расходах и доходах домохозяйств.
В рамках работы данные обследований описываются следующими показателями:
- код территории по ОКАТО (переменная \textbf{ТЕРРИТОРИЯ})\\
- тип населенного пункта (переменная \textbf{ТИПНАС}, 1 - город, 2 - село)\\
- расходы на продукты питания (переменная \textbf{ПРОДПИТ}, \emph{y}(1))\\
- расходы на непродовольственные товары (переменная \textbf{НЕПРОД}, \emph{y}(2))\\
- расходы на оплату услуг (переменная \textbf{УСЛУГИ}, \emph{y}(3))\\
- расходы на покупку алкогольной продукции (переменная \textbf{АЛКО}, \emph{x}\textsubscript{\emph{0}})\\
- доходы домохозяйства (переменная \textbf{ДОХОД}, \emph{x}\textsubscript{\emph{1}})\\
- сбережения домохозяйства (переменная \textbf{СБЕРЕЖ}, \emph{x}\textsubscript{\emph{2}})\\
- число людей в домохозяйстве (\textbf{ЧИСЛОЛЮД}, \emph{x}\textsubscript{\emph{3}})\\
- число детей в домохозяйстве (\textbf{ЧИСЛОДЕТ}, \emph{x}\textsubscript{\emph{4}})

Федеральный округ и отклик \emph{y} берутся в соответствии с выданным вариантом.

\hypertarget{ux43fux43eux440ux44fux434ux43eux43a-ux432ux44bux43fux43eux43bux43dux435ux43dux438ux44f-ux440ux430ux431ux43eux442ux44b}{%
\section{Порядок выполнения работы}\label{ux43fux43eux440ux44fux434ux43eux43a-ux432ux44bux43fux43eux43bux43dux435ux43dux438ux44f-ux440ux430ux431ux43eux442ux44b}}

\textbf{1.} Прочитать исходные данные, Исследовать структуру данных: сколько наблюдений, какие поля, какого типа, есть ли пропуски. Исключить отклики \emph{y}, не соответствующие выданному варианту.\\
\textbf{2.} Дополнить данные полями из файла \emph{codes.csv}:\\
- \textbf{Код}: код региона по ОКАТО, соответствующий переменной ТЕРРИТОРИЯ в данных по обследованиям\\
- \textbf{Название}: название региона\\
- \textbf{ФО}: федеральный округ, к которому относится регион\\
- \textbf{hc-a2}: код региона, соответствующий переменной hc-a2 карты ``Russia with disputed territories'' библиотеки Highcharter

Сколько наблюдений приходится на каждый из федеральных округов?\\
Вывести топ-10 регионов по убыванию количества наблюдений.\\
Вывести топ-10 регионов по убыванию среднего значения отклика \emph{y}.

\textbf{3.} Рассчитать новую переменную: доход на одного взрослого (\emph{x}\textsubscript{\emph{1}}/(\emph{x}\textsubscript{\emph{3}}-\emph{x}\textsubscript{\emph{4}})).\\
По полученному показателю Для каждого региона рассчитать среднее арифметическое и медиану.\\
Отобразить полученные значения на карте ``Russia with disputed territories'' библиотеки Highcharter (с учётом, что в данных по обследованиям ХМАО и ЯНАО относятся к Тюменской области, а Ненецкий АО к Архангельской области). Цветовая шкала в зависимости от среднего, во всплывающем окне выводить название рениона, среднее и медиану.

Проанализировать результаты, сделать выводы. Как различаются регионы между собой? Отличается ли средний доход от медианного дохода. В связи с чем возникают различия?

\textbf{4.} Отфильтровать данные, оставив наблюдения по федеральному округу в соответствии с вариантом. Отклик
На одном рисунке построить корреляционные поля, рассчитать коэффициенты корреляции и распределения между переменными \emph{y}, \emph{x}\textsubscript{\emph{0}}, \emph{x}\textsubscript{\emph{1}}, \emph{x}\textsubscript{\emph{2}}, \emph{x}\textsubscript{\emph{3}}, \emph{x}\textsubscript{\emph{4}} (функция ggpairs() библиотеки GGally). Сделать выводы о наличии связи между показателями, её виде и форме.

\textbf{5.} Провести проверку данных на наличие выбросов с удалением всех наблюдений \(\overline{x}_{(i)}=(x_{1i},x_{2i})\), для которых выполняется соотношение \(h_{i}=\overline{x}_{(i)}(X^{T}X)^{-1}\overline{x}_{(i)}^{T}\geqslant \frac{3}{N}\), где \(X\) - матрица, составленная из значений переменных \emph{x}\textsubscript{\emph{1}} и \emph{x}\textsubscript{\emph{2}}. На основе оставшихся наблюдений провести расчет основных выборочных характеристик, проинтерпретировать результаты, сделать выводы.

\textbf{6.} Рассчитать выборочные парные коэффициенты корреляции для всех возможных пар переменных, проверить их на значимость, сделать выводы о тесноте связи между признаками.

\textbf{7.} Построить корреляционное поле между откликом и доходом.\\
Построить линейную модель парной регрессии между этими переменными. Добавить линию регрессии на координатную плоскость. Проанализировать результаты (форма корреляционного поля, что показывают оценки параметров, каково качество модели, выводы о значимости параметров и модели).

Проделать то же самое по прологарифмированным переменным. Сделать выводы, что изменилось?

Доп. баллы: для двух моделей построить и проанализировать корреляционное поле между доходом и модулем остатков модели. С помощию критерия Голдфельда-Квандта изучить обе модели на наличие гетероскедастичности. Сделать выводы.

\textbf{8.} В модель с логарифмированными переменными добавить факторную переменную ТИПНАС, учитывая взаимодействие этой переменной с величиной дохода. Построить корреляционное поле, новую переменную обозначив цветом. Добавить линии регрессии, проинтерпретировать результаты и сделать выводы.

\textbf{9.} Построить линейную модель множественной регресии по всем входным переменным \emph{x}\textsubscript{\emph{0}}, \emph{x}\textsubscript{\emph{1}}, \emph{x}\textsubscript{\emph{2}}, \emph{x}\textsubscript{\emph{3}}, \emph{x}\textsubscript{\emph{4}}, сделать выводы. Проранжировать входные факторы по степени влияния на отклик при помощи коэффициентов эластичности. Отобрать два наиболее сильно влияющих фактора, изобразить в трёх измерениях модель частной регрессии с этими факторами, заменив медианами значения остальных переменных: добавить исходные данные, плоскость регрессии и плоскости 95\% доверительного интервала. Затем построить этот график, прологарифмировав значения отклика, прогнозов и доверительных интервалов. Сделать выводы.

\textbf{10.} Сделать общие выводы по проделанной работе в терминах исходных показателей.

\hypertarget{intro}{%
\chapter{Начало работы}\label{intro}}

\hypertarget{ux443ux441ux442ux430ux43dux43eux432ux43aux430-r-ux438-rstudio}{%
\section{Установка R и Rstudio}\label{ux443ux441ux442ux430ux43dux43eux432ux43aux430-r-ux438-rstudio}}

Скачиваем установочный файл R \href{https://cran.r-project.org/bin/windows/base/}{по ссылке}. (кнопка ``Download R 3.6.3 for Windows'' или более поздняя версия). Запускаем, устанавливаем без изменений.

Скачиваем и устанавливаем RStudio \href{https://rstudio.com/products/rstudio/download/\#download}{с официального сайта}. RStudio - среда разработки для языка R, позволяющая работать более комфортно.

Запускаем RStudio:\\
\includegraphics{rgr_econometrics_files/2020-04-11 06_58_25-RStudio.png}

  \bibliography{book.bib,packages.bib}

\end{document}
